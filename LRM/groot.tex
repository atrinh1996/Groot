\documentclass[12pt]{article}
\usepackage[utf8]{inputenc}
\usepackage{parskip}
\usepackage{tabularx}

%%%%%%%%%%%%%%%%%%%%%%%%%%%%%%%%%%%%%%%%%%%%%%%%%%%%%%%%%%%%%%%%%%%%%%%%
% Reduce margin
%
% \addtolength{\oddsidemargin}{-.85in}
% \addtolength{\evensidemargin}{-.85in}
% \addtolength{\textwidth}{1in}

% \addtolength{\topmargin}{-.85in}
% \addtolength{\textheight}{1in}

% Page format commands:
% Override normal article margins,
% making the margins smaller
\setlength{\textwidth}{6.5in}
\setlength{\textheight}{9in}
\setlength{\oddsidemargin}{0in}
\setlength{\evensidemargin}{0in}
\setlength{\topmargin}{-0.5in}

\setlength{\parindent}{0pt}
%%%%%%%%%%%%%%%%%%%%%%%%%%%%%%%%%%%%%%%%%%%%%%%%%%%%%%%%%%%%%%%%%%%%%%%%


%%%%%%%%%%%%%%%%%%%%%%%%%%%%%%%%%%%%%%%%%%%%%%%%%%%%%%%%%%%%%%%%%%%%%%%%
% Math Symbols
\usepackage{mathtools}
\usepackage{amssymb}
% \usepackage{epsfig}
\usepackage{amsmath,amsthm}
\usepackage{amscd,amsxtra,latexsym}


% add floor and ceiling symbol. Usage: \ceil*{}, \floor*{}
% \DeclarePairedDelimiter\ceil{\lceil}{\rceil}
% \DeclarePairedDelimiter\floor{\lfloor}{\rfloor}

% multiset \langle ... \rangle
\def\multiset#1#2{\ensuremath{\left(\kern-.3em\left(\genfrac{}{}{0pt}{}{#1}{#2}\right)\kern-.3em\right)}}



%%%%%%%%%%%%%%%%%%%%%%%%%%%%%%%%%%%%%%%%%%%%%%%%%%%%%%%%%%%%%%%%%%%%%%%%


%%%%%%%%%%%%%%%%%%%%%%%%%%%%%%%%%%%%%%%%%%%%%%%%%%%%%%%%%%%%%%%%%%%%%%%%
% Code Sample Styling

% use \lstinline! (inline text) ! or \begin{lstlisting} text block \end{lstlisting}
\usepackage{listings}

\usepackage{color}
\definecolor{light-gray}{gray}{0.97} % shade of grey
\definecolor{dkgreen}{rgb}{0,0.6,0}
\definecolor{gray}{rgb}{0.5,0.5,0.5}
\definecolor{mauve}{rgb}{0.58,0,0.82}

% \begin{lstlisting}[...] ... \end{lstlisting}
\lstset{frame=none,
    language=LISP,
    aboveskip=3mm,
    belowskip=3mm,
    stepnumber=0, % set to 0 if you don't like line nums
    showstringspaces=false,
    columns=flexible,
    basicstyle={\small\ttfamily},
    numbers=left,
    numberstyle=\color{black},
    keywordstyle=\color{blue},
    commentstyle=\color{dkgreen},
    stringstyle=\color{mauve},
    backgroundcolor=\color{light-gray},
    breaklines=true,
    breakatwhitespace=false,
    tabsize=2
}



%%%%%%%%%%%%%%%%%%%%%%%%%%%%%%%%%%%%%%%%%%%%%%%%%%%%%%%%%%%%%%%%%%%%%%%%


%%%%%%%%%%%%%%%%%%%%%%%%%%%%%%%%%%%%%%%%%%%%%%%%%%%%%%%%%%%%%%%%%%%%%%%%
\usepackage{xcolor}
%% https://tex.stackexchange.com/questions/401750/quick-and-short-command-for-coloring-one-word
\newcommand\shorthandon{\catcode`@=\active \catcode`^=\active \catcode`*=\active }
\newcommand\shorthandoff{\catcode`@=12 \catcode`^=7 \catcode`*=12 }
\shorthandon
\def@#1@{\textcolor{red}{#1}}%
\def^#1^{\textcolor{blue}{#1}}%
\def*#1{\string#1}
\shorthandoff
%% useage: \textcolor{red}{text here}
% \shorthandon
% This is a @test@ of the ^emergency^ bro*@dcast system.
% \shorthandoff
%%%%%%%%%%%%%%%%%%%%%%%%%%%%%%%%%%%%%%%%%%%%%%%%%%%%%%%%%%%%%%%%%%%%%%%%


%%%%%%%%%%%%%%%%%%%%%%%%%%%%%%%%%%%%%%%%%%%%%%%%%%%%%%%%%%%%%%%%%%%%%%%%
% Misc
\usepackage{graphicx} % graphics
\usepackage{enumitem} % listing style (bullet lists)

% below helps with trying to get figures in a row
% \usepackage{caption}
% \usepackage{subcaption}

% hyperlink styling
% use \href{} and \url{}, and colors table of contents links
% use \href{} and \url{}
% \label{sec:name}
% \hyperref[label]{text}
\usepackage{hyperref}
\hypersetup{
    colorlinks=true,
    linkcolor=blue, % was previously black
    filecolor=magenta,
    urlcolor=blue,
    pdftitle={groot LRM}
}
\urlstyle{same}

% A command for primes (')
\newcommand{\p}%
    {\ensuremath{^{\prime}}}

% a command for double primes ('')
\newcommand{\pp}%
    {\ensuremath{^{\prime \prime}}}

% A command for the Kleene star
\newcommand{\str}%
    {\ensuremath{^{\star}}}

% a command for the double star
\newcommand{\sstr}%
    {\ensuremath{^{\star\star}}}
%%%%%%%%%%%%%%%%%%%%%%%%%%%%%%%%%%%%%%%%%%%%%%%%%%%%%%%%%%%%%%%%%%%%%%%%



\begin{document}
%%%%%%%%%%%%%%%%%%%%%%%%%%%%%%%%%%%%%%%%%%%%%%%%%%%%%%%%%%%%%%%%%%%%%%%%
%%%% TITLE PAGE
%%%%%%%%%%%%%%%%%%%%%%%%%%%%%%%%%%%%%%%%%%%%%%%%%%%%%%%%%%%%%%%%%%%%%%%%

\begin{titlepage}
    \title{(g)ROOT \\ Language Reference Manual}
    \author{Samuel Russo \quad Amy Bui \quad Eliza Encherman \\ Zachary Goldstein \quad Nickolas Gravel}
    \date{\today}
    \maketitle
    \begin{figure}[h]
        \centering
        \includegraphics[width=0.35\linewidth]{images/baby_groot.PNG}
        \label{fig:im_groot_title}
    \end{figure}
\end{titlepage}


%%%%%%%%%%%%%%%%%%%%%%%%%%%%%%%%%%%%%%%%%%%%%%%%%%%%%%%%%%%%%%%%%%%%%%%%
% Table of Contents
\setcounter{tocdepth}{2}
\tableofcontents
\pagebreak 
%%%%%%%%%%%%%%%%%%%%%%%%%%%%%%%%%%%%%%%%%%%%%%%%%%%%%%%%%%%%%%%%%%%%%%%%


%%%%%%%%%%%%%%%%%%%%%%%%%%%%%%%%%%%%%%%%%%%%%%%%%%%%%%%%%%%%%%%%%%%%%%%%
\section{This is a section heading with number}
\label{sec:start} % section label, use to link withing file
                    % \hyperref[sec:start]{text}

This is a bullet list:
\begin{itemize}
    \item Link to \href{https://www.overleaf.com/learn/latex/Hyperlinks}{Overleaf href}
    \item url example: \url{https://canvas.tufts.edu/}
    \item link to aanother labeled section: \hyperref[sec:hello]{Hello World section}
\end{itemize}

This is a enumlist (default 1):
\begin{enumerate}
    \item Inline code: \lstinline!(val x 42)! example.
    \item \textsc{Hello World}
    \item \texttt{Hello World}
    \item \textsf{Hello World}
    \item This is \textcolor{red}{very Red}
    \item This is \textcolor{blue}{Bluey}
\end{enumerate}

This is enumlist with alph bullets:
\begin{enumerate}[label=\alph*.]
    \item $x^2 = y$
    \item x = y
\end{enumerate}


%% start on new page
\pagebreak
%%%%%%%%%%%%%%%%%%%%%%%%%%%%%%%%%%%%%%%%%%%%%%%%%%%%%%%%%%%%%%%%%%%%%%%%


%%%%%%%%%%%%%%%%%%%%%%%%%%%%%%%%%%%%%%%%%%%%%%%%%%%%%%%%%%%%%%%%%%%%%%%%
\section{Hello World}
\label{sec:hello}

    \subsection{Hello Subsection example}
        This is how you include pictures:

        % \begin{figure}[h]
        %     \centering
        %     \includegraphics[width=0.7\linewidth]{FILE_PATH}
        %     % \caption{} %% optional caption
        %     \label{fig:FILE_NAME}
        % \end{figure}

        \begin{figure}[h]
            \centering
            \includegraphics[width=0.2\linewidth]{images/baby_groot.PNG}
            \caption{Our little mascot}
            \label{fig:im_groot}
        \end{figure}


        \subsubsection{Basic Table: This doesn't appeear in contents}
        \begin{center}
            \begin{tabular}{|r|c|l|}
                \hline
                hrow1 & hrow2 & hrow3 \\
                \hline
                This is & This is & This is\\
                right-aligned & centered & left aligned \\
                \hline
            \end{tabular}
        \end{center}

        \begin{center}
            \begin{tabular}{l|l|l}
                $\leftarrow$ & $\Leftarrow$ & $\langle x + y \rangle$ \\ 
                $\rightarrow$ & $\Rightarrow$ & $| x |$ \\ 
                $\downarrow$ & $\Downarrow$ & $\| x^2 \|$ \\
                $\leftrightarrow$ & $\uparrow$ & \Bigg( \big( cs107 \big)  \Bigg) \\ 
                $\epsilon$ & $\rho$ & \Bigg[ \big[ cs107 \big] \Bigg] \\
                $\Sigma$ & $\eta$ & \\ 
                $\mu$ & \{ \} & \\
            \end{tabular}
        \end{center}

    \subsection{Another hello subsection example}

        \subsubsection{Coding environment without line numbers (default)}
\begin{lstlisting}
(val add
(lambda (x y) 
    (+ x y)))
\end{lstlisting}

        \subsubsection{Coding environment with line numbers (this can be made default)}
\begin{lstlisting}[stepnumber=1]
(val add
(lambda (x y) 
    (+ x y)))
\end{lstlisting}

        \subsubsection{Code side-by-side}
\begin{minipage}[h]{0.5\textwidth}
\begin{lstlisting}
(val who 
(lambda (i am groot) #t)
)
\end{lstlisting}
\end{minipage}
\begin{minipage}[h]{0.5\textwidth}
\begin{lstlisting}
(val were
(lambda (senza) (> 3 0))
)
\end{lstlisting}
\end{minipage}
\pagebreak
%%%%%%%%%%%%%%%%%%%%%%%%%%%%%%%%%%%%%%%%%%%%%%%%%%%%%%%%%%%%%%%%%%%%%%%%


%% LRM OUTLINE STARTS HERE
%%%%%%%%%%%%%%%%%%%%%%%%%%%%%%%%%%%%%%%%%%%%%%%%%%%%%%%%%%%%%%%%%%%%%%%%
\section{Intro}
\label{sec:intro}

\href{https://ocaml.org/manual/language.html}{Ocaml LRM}

\pagebreak
%% End of Introduction %%%%%%%%%%%%%%%%%%%%%%%%%%%%%%%%%%%%%%%%%%%


%%%%%%%%%%%%%%%%%%%%%%%%%%%%%%%%%%%%%%%%%%%%%%%%%%%%%%%%%%%%%%%%%%%%%%%%
\section{Lexical Convention}
\label{sec:lexcon}

    \subsection{Blanks}

    \subsection{Comments}

    \subsection{Identifiers}

    \subsection{Integer Literals}

    \subsection{Boolean Literals}

    \subsection{Character Literals}

    \subsection{Operators}

    \subsection{Keywords}

% \pagebreak
%% End of Lexical Convention %%%%%%%%%%%%%%%%%%%%%%%%%%%%%%%%%%%%%%%%%%%


%%%%%%%%%%%%%%%%%%%%%%%%%%%%%%%%%%%%%%%%%%%%%%%%%%%%%%%%%%%%%%%%%%%%%%%%
\section{Values}
\label{sec:values}

    \subsection{Base Values}

    \subsection{Functions}

% \pagebreak
%% End of Values %%%%%%%%%%%%%%%%%%%%%%%%%%%%%%%%%%%%%%%%%%%%%%%%%%%%%%



%%%%%%%%%%%%%%%%%%%%%%%%%%%%%%%%%%%%%%%%%%%%%%%%%%%%%%%%%%%%%%%%%%%%%%%%
\section{Names}
\label{sec:names}

    \subsection{Base Values}

    \subsection{Functions}

% \pagebreak
%% End of Names %%%%%%%%%%%%%%%%%%%%%%%%%%%%%%%%%%%%%%%%%%%



%%%%%%%%%%%%%%%%%%%%%%%%%%%%%%%%%%%%%%%%%%%%%%%%%%%%%%%%%%%%%%%%%%%%%%%%
\section{Constants}
\label{sec:const}


% \pagebreak
%% End of Constants %%%%%%%%%%%%%%%%%%%%%%%%%%%%%%%%%%%%%%%%%%%



%%%%%%%%%%%%%%%%%%%%%%%%%%%%%%%%%%%%%%%%%%%%%%%%%%%%%%%%%%%%%%%%%%%%%%%%
\section{Expressions}
\label{sec:expr}


% \pagebreak
%% End of Expressions %%%%%%%%%%%%%%%%%%%%%%%%%%%%%%%%%%%%%%%%%%%



\end{document}