\documentclass[12pt]{article}
\usepackage[utf8]{inputenc}
\usepackage{parskip}
\usepackage{tabularx}
\usepackage{syntax} % grammar format
% \usepackage{markdown} % inline mark down
\usepackage[color, leftbars]{changebar}

%%%%%%%%%%%%%%%%%%%%%%%%%%%%%%%%%%%%%%%%%%%%%%%%%%%%%%%%%%%%%%%%%%%%%%%%
% Reduce margin
%
% \addtolength{\oddsidemargin}{-.85in}
% \addtolength{\evensidemargin}{-.85in}
% \addtolength{\textwidth}{1in}

% \addtolength{\topmargin}{-.85in}
% \addtolength{\textheight}{1in}

% Page format commands:
% Override normal article margins,
% making the margins smaller
\setlength{\textwidth}{6.5in}
\setlength{\textheight}{9in}
\setlength{\oddsidemargin}{0in}
\setlength{\evensidemargin}{0in}
\setlength{\topmargin}{-0.5in}

\setlength{\parindent}{0pt}

\setlength{\grammarparsep}{20pt plus 1pt minus 1pt} % increase separation between rules
\setlength{\grammarindent}{10em} % increase separation between LHS/RHS 
%%%%%%%%%%%%%%%%%%%%%%%%%%%%%%%%%%%%%%%%%%%%%%%%%%%%%%%%%%%%%%%%%%%%%%%%


%%%%%%%%%%%%%%%%%%%%%%%%%%%%%%%%%%%%%%%%%%%%%%%%%%%%%%%%%%%%%%%%%%%%%%%%
% Math Symbols
\usepackage{mathtools}
\usepackage{amssymb}
% \usepackage{epsfig}
\usepackage{amsmath,amsthm}
\usepackage{amscd,amsxtra,latexsym}


% add floor and ceiling symbol. Usage: \ceil*{}, \floor*{}
% \DeclarePairedDelimiter\ceil{\lceil}{\rceil}
% \DeclarePairedDelimiter\floor{\lfloor}{\rfloor}

% multiset \langle ... \rangle
\def\multiset#1#2{\ensuremath{\left(\kern-.3em\left(\genfrac{}{}{0pt}{}{#1}{#2}\right)\kern-.3em\right)}}



%%%%%%%%%%%%%%%%%%%%%%%%%%%%%%%%%%%%%%%%%%%%%%%%%%%%%%%%%%%%%%%%%%%%%%%%


%%%%%%%%%%%%%%%%%%%%%%%%%%%%%%%%%%%%%%%%%%%%%%%%%%%%%%%%%%%%%%%%%%%%%%%%
% Code Sample Styling

% use \lstinline! (inline text) ! or \begin{lstlisting} text block \end{lstlisting}
\usepackage{listings}

\usepackage{color}
\definecolor{light-gray}{gray}{0.97} % shade of grey
\definecolor{dkgreen}{rgb}{0,0.6,0}
\definecolor{gray}{rgb}{0.5,0.5,0.5}
\definecolor{mauve}{rgb}{0.58,0,0.82}

% \begin{lstlisting}[...] ... \end{lstlisting}
\lstset{frame=none,
    % language=LISP,
    aboveskip=3mm,
    belowskip=3mm,
    stepnumber=0, % set to 0 if you don't like line nums
    showstringspaces=false,
    columns=flexible,
    basicstyle={\small\ttfamily},
    numbers=left,
    numberstyle=\color{black},
    keywordstyle=\color{blue},
    commentstyle=\color{dkgreen},
    stringstyle=\color{mauve},
    backgroundcolor=\color{light-gray},
    breaklines=true,
    breakatwhitespace=false,
    tabsize=2
}



%%%%%%%%%%%%%%%%%%%%%%%%%%%%%%%%%%%%%%%%%%%%%%%%%%%%%%%%%%%%%%%%%%%%%%%%


%%%%%%%%%%%%%%%%%%%%%%%%%%%%%%%%%%%%%%%%%%%%%%%%%%%%%%%%%%%%%%%%%%%%%%%%
\usepackage{xcolor}
%% https://tex.stackexchange.com/questions/401750/quick-and-short-command-for-coloring-one-word
\newcommand\shorthandon{\catcode`@=\active \catcode`^=\active \catcode`*=\active }
\newcommand\shorthandoff{\catcode`@=12 \catcode`^=7 \catcode`*=12 }
\shorthandon
\def@#1@{\textcolor{red}{#1}}%
\def^#1^{\textcolor{blue}{#1}}%
\def*#1{\string#1}
\shorthandoff
%% useage: \textcolor{red}{text here}
% \shorthandon
% This is a @test@ of the ^emergency^ bro*@dcast system.
% \shorthandoff
%%%%%%%%%%%%%%%%%%%%%%%%%%%%%%%%%%%%%%%%%%%%%%%%%%%%%%%%%%%%%%%%%%%%%%%%


%%%%%%%%%%%%%%%%%%%%%%%%%%%%%%%%%%%%%%%%%%%%%%%%%%%%%%%%%%%%%%%%%%%%%%%%
% Misc
\usepackage{graphicx} % graphics
\usepackage{enumitem} % listing style (bullet lists)

% below helps with trying to get figures in a row
% \usepackage{caption}
% \usepackage{subcaption}

% hyperlink styling
% use \href{} and \url{}, and colors table of contents links
% use \href{} and \url{}
% \label{sec:name}
% \hyperref[label]{text}
\usepackage{hyperref}
\hypersetup{
    colorlinks=true,
    linkcolor=blue, % was previously black
    filecolor=magenta,
    urlcolor=blue,
    pdftitle={groot LRM}
}
\urlstyle{same}

% A command for primes (')
\newcommand{\p}%
    {\ensuremath{^{\prime}}}

% a command for double primes ('')
\newcommand{\pp}%
    {\ensuremath{^{\prime \prime}}}

% A command for the Kleene star
\newcommand{\str}%
    {\ensuremath{^{\star}}}

% a command for the double star
\newcommand{\sstr}%
    {\ensuremath{^{\star\star}}}
%%%%%%%%%%%%%%%%%%%%%%%%%%%%%%%%%%%%%%%%%%%%%%%%%%%%%%%%%%%%%%%%%%%%%%%%



\begin{document}
\title{(g)ROOT \\ Language Reference Manual}
\author{Samuel Russo \quad Amy Bui \quad Eliza Encherman \\ Zachary Goldstein \quad Nickolas Gravel}
\date{\today}
\maketitle

    %%%%%%%%%%%%%%%%%%%%%%%%%%%%%%%%%%%%%%%%%%%%%%%%%%%%%%%%%%%%%%%%%%%%%%%%
    % Table of Contents
    \setcounter{tocdepth}{2}
    \tableofcontents
    \pagebreak 
    %%%%%%%%%%%%%%%%%%%%%%%%%%%%%%%%%%%%%%%%%%%%%%%%%%%%%%%%%%%%%%%%%%%%%%%%

    %%%%%%%%%%%%%%%%%%%%%%%%%%%%%%%%%%%%%%%%%%%%%%%%%%%%%%%%%%%%%%%%%%%%%%%%
    % \section{This is a section heading with number}
    % \label{sec:start} % section label, use to link withing file
    %                   % \hyperref[sec:start]{text}

    % This is a bullet list:
    % \begin{itemize}
    %     \item Link to \href{https://www.overleaf.com/learn/latex/Hyperlinks}{Overleaf href}
    %     \item url example: \url{https://canvas.tufts.edu/}
    %     \item link to aanother labeled section: \hyperref[sec:hello]{Hello World section}
    % \end{itemize}

    % This is a enumlist (default 1):
    % \begin{enumerate}
    %     \item Inline code: \lstinline!(val x 42)! example.
    %     \item \textsc{Hello World}
    %     \item \texttt{Hello World}
    %     \item \textsf{Hello World}
    %     \item This is \textcolor{red}{very Red}
    %     \item This is \textcolor{blue}{Bluey}
    % \end{enumerate}

    % This is enumlist with alph bullets:
    % \begin{enumerate}[label=\alph*.]
    %     \item $x^2 = y$
    %     \item x = y
    % \end{enumerate}

	% \subsection{Grammar}


	% \begin{grammar}
		
	% 	<statement> ::= <ident> `=' <expr> 
	% 	\alt `for' <ident> `=' <expr> `to' <expr> `do' <statement> 
	% 	\alt `{' <stat-list> `}' 
	% 	\alt <empty> 
		
	% 	<stat-list> ::= <statement> `;' <stat-list> | <statement> 
		
	% \end{grammar}


    % %% start on new page
    % \pagebreak
    % %%%%%%%%%%%%%%%%%%%%%%%%%%%%%%%%%%%%%%%%%%%%%%%%%%%%%%%%%%%%%%%%%%%%%%%%


    % %%%%%%%%%%%%%%%%%%%%%%%%%%%%%%%%%%%%%%%%%%%%%%%%%%%%%%%%%%%%%%%%%%%%%%%%
    % \section{Hello World}
%     \label{sec:hello}

%         \subsection{Hello Subsection example}
%             This is how you include pictures:

%             % \begin{figure}[h]
%             %     \centering
%             %     \includegraphics[width=0.7\linewidth]{FILE_PATH}
%             %     % \caption{} %% optional caption
%             %     \label{fig:FILE_NAME}
%             % \end{figure}

%             % \begin{figure}[h]
%             %     \centering
%             %     \includegraphics[width=0.2\linewidth]{/images/baby_groot.PNG}
%             %     \caption{Our little mascot}
%             %     \label{fig:im_groot}
%             % \end{figure}
            
%             \begin{figure}[h]
%             	\centering
%             	\includegraphics[width=0.2\linewidth]{images/gt}
%             	\caption{}
%             	\label{fig:babygroot}
%             \end{figure}
            


%             \subsubsection{Basic Table: This doesn't appeear in contents}
%             \begin{center}
%                 \begin{tabular}{|r|c|l|}
%                     \hline
%                     hrow1 & hrow2 & hrow3 \\
%                     \hline
%                     This is & This is & This is\\
%                     right-aligned & centered & left aligned \\
%                     \hline
%                 \end{tabular}
%             \end{center}

%             \begin{center}
%                 \begin{tabular}{l|l|l}
%                     $\leftarrow$ & $\Leftarrow$ & $\langle x + y \rangle$ \\ 
%                     $\rightarrow$ & $\Rightarrow$ & $| x |$ \\ 
%                     $\downarrow$ & $\Downarrow$ & $\| x^2 \|$ \\
%                     $\leftrightarrow$ & $\uparrow$ & \Bigg( \big( cs107 \big)  \Bigg) \\ 
%                     $\epsilon$ & $\rho$ & \Bigg[ \big[ cs107 \big] \Bigg] \\
%                     $\Sigma$ & $\eta$ & \\ 
%                     $\mu$ & \{ \} & \\
%                 \end{tabular}
%             \end{center}

%         \subsection{Another hello subsection example}

%             \subsubsection{Coding environment without line numbers (default)}
% \begin{lstlisting}
% (val add
%     (lambda (x y) 
%         (+ x y)))
% \end{lstlisting}

%             \subsubsection{Coding environment with line numbers (this can be made default)}
% \begin{lstlisting}[stepnumber=1]
% (val add
%     (lambda (x y) 
%         (+ x y)))
% \end{lstlisting}

%             \subsubsection{Code side-by-side}
% \begin{minipage}[h]{0.5\textwidth}
% \begin{lstlisting}
% (val who 
%     (lambda (i am groot) #t)
% )
% \end{lstlisting}
% \end{minipage}
% \begin{minipage}[h]{0.5\textwidth}
% \begin{lstlisting}
% (val were
%     (lambda (senza) (> 3 0))
% )
% \end{lstlisting}
% \end{minipage}
%     \pagebreak
%     %%%%%%%%%%%%%%%%%%%%%%%%%%%%%%%%%%%%%%%%%%%%%%%%%%%%%%%%%%%%%%%%%%%%%%%%


%% LRM OUTLINE STARTS HERE
%%%%%%%%%%%%%%%%%%%%%%%%%%%%%%%%%%%%%%%%%%%%%%%%%%%%%%%%%%%%%%%%%%%%%%%%
\section{Intro}
\label{sec:intro}

\href{https://ocaml.org/manual/language.html}{Ocaml LRM}

    \subsection{How to read manual}
    The syntax of the language will be given in BNF-like notation. Non-terminal symbol will be in italic font \emph{like-this}, square brackets [ … ] denote optional components, curly braces \{ … \} denote zero or more repetitions of the enclosed component,  and parenthesis ( … ) denote a grouping.

    \pagebreak
    %% End of Introduction %%%%%%%%%%%%%%%%%%%%%%%%%%%%%%%%%%%%%%%%%%%


    %%%%%%%%%%%%%%%%%%%%%%%%%%%%%%%%%%%%%%%%%%%%%%%%%%%%%%%%%%%%%%%%%%%%%%%%
    \section{Lexical Convention}
    \label{sec:lexcon}
        
        \subsection{Blanks} %%%
        The following characters are considered as \textbf{blanks}: space, horizontal tab (`\texttt{\textbackslash t}'), newline character (`\texttt{\textbackslash n}'), and carriage return (`\texttt{\textbackslash r}'). 

        Blanks separate adjacent identifiers, literals, expressions, and keywords. They are otherwise ignored.
        %%%%%%%%%%%%%%%%%%%%%%%%%%%%%%%%%%%%%%%%%%%%%%%%%%%%%%%%%%%%%%%%%%%%%%%%

        \subsection{Comments}
        Comments are introduced with two adjact characters \texttt{(;} and terminated by two adjacent characters \texttt{;)}. Nested comments are currently not allowed.\\

% \hspace{1cm}\begin{minipage}[h]{1\textwidth}
\cbcolor{green}
\cbstart
\begin{lstlisting}
(; This is a comment. ;)
\end{lstlisting}
\cbend
% \end{minipage}
        %%%%%%%%%%%%%%%%%%%%%%%%%%%%%%%%%%%%%%%%%%%%%%%%%%%%%%%%%%%%%%%%%%%%%%%%

        \subsection{Identifiers}
        \label{subsec:id}
        Identifiers are sequences of letters, digits, and underscore characters (`\_'), starting with a letter. Letters consist of the 26 lowercase and 26 uppercase characters from the ASCII set.

        \hspace{1cm}\begin{minipage}[h]{1\textwidth}
        \begin{grammar}
            <ident> ::= \texttt{$letter$\ (\ $letter$ | $digit$ | \_ \ )}
            
            <letter> ::= \texttt{a...z\ |\ A...Z }

            <digit> ::= \texttt{0...9 }
        \end{grammar}
        \end{minipage}
        %%%%%%%%%%%%%%%%%%%%%%%%%%%%%%%%%%%%%%%%%%%%%%%%%%%%%%%%%%%%%%%%%%%%%%%%


        \subsection{Integer Literals}
        \label{subsec:intlit}
        An integer literal is a decimal, represented by a sequence of one or more digits, optionally preceded by a minus sign. 

            \hspace{1cm}\begin{minipage}[h]{1\textwidth}
            \begin{grammar}
                <integer-literal> ::= \texttt{[-] $digit$ \{ $digit$ \}}
    
                <digit> ::= \texttt{0...9 }
            \end{grammar}
            \end{minipage}
        %%%%%%%%%%%%%%%%%%%%%%%%%%%%%%%%%%%%%%%%%%%%%%%%%%%%%%%%%%%%%%%%%%%%%%%%

        \subsection{Boolean Literals}
        \label{subsec:boollit}
        Boolean literals are represented by two adjacent characters; the first is the octothorp character (\texttt{\#}), and it is immediately followed by either the \texttt{t} or the \texttt{f} character. 

            \hspace{1cm}\begin{minipage}[h]{1\textwidth}
            \begin{grammar}
                <boolean-literal> ::= \texttt{\# (\ t |\ f\ )}
            \end{grammar}
            \end{minipage}
        %%%%%%%%%%%%%%%%%%%%%%%%%%%%%%%%%%%%%%%%%%%%%%%%%%%%%%%%%%%%%%%%%%%%%%%%

        \subsection{Character Literals}
        \label{subsec:charlit}
        Character literals are a single character enclosed by two \textsf{'} (single-quote) characters.

        %%%%%%%%%%%%%%%%%%%%%%%%%%%%%%%%%%%%%%%%%%%%%%%%%%%%%%%%%%%%%%%%%%%%%%%%

        \subsection{Operators}
        \label{subsec:op}
        All of the following operators are prefix characters or prefixed characters read as single token. Binary operators are expected to be followed by two expressions, unary operatprs are expected to be followed by one expression. 

        \hspace{1cm}\begin{minipage}[h]{1\textwidth}
            \begin{grammar}

                <operator> ::= (\ $unary-operator$ |\ $binary-operator$\ )

                <unary-operator> ::= \texttt{!}

                <binary-operator> ::= \ \texttt{+\ |\ -\ |\ *\ |\ /\ |\ mod }
                \alt\texttt{==\ |\ \textless\ |\ \textgreater\ |\ $\leq$ |\ $\geq$\ |\ != }
                \alt\texttt{\&\&\ |\ \textcolor{red}{||} }

            \end{grammar}
            \end{minipage}

        \subsection{Keywords}
        \label{subsec:key}
        The below identifiers are reserved keywords and cannot be used otherwise: \\

\cbcolor{green}
\cbstart
\begin{lstlisting}
if      val         let
leaf?   elm         tree
cld     sib         lambda
\end{lstlisting}
\cbend

        The following character sequence are also keywords: \\

\cbcolor{green}
\cbstart
\begin{lstlisting}
==      +       &&      >       '
!=      -       ||      mod     #t
<=      *       !       (       #f
>=      /       <       )
\end{lstlisting}
\cbend

    \pagebreak
    %% End of Lexical Convention %%%%%%%%%%%%%%%%%%%%%%%%%%%%%%%%%%%%%%%%%%%


%%%%%%%%%%%%%%%%%%%%%%%%%%%%%%%%%%%%%%%%%%%%%%%%%%%%%%%%%%%%%%%%%%%%%%%%
\section{Values}
\label{sec:values}

    \subsection{Base Values}

            \subsubsection{Integer numbers}
            Integer values are integer numbers in range from $-2^{32}$ to $2^{32} - 1$, similar to LLVM's integers, and may support a wider range of integer values on other machines, such as $-2^{64}$ to $2^{64} - 1$ on a 64-bit machine.

            \subsubsection{Boolean values}
            Booleans have two values. \texttt{\#t} evaluates to the boolean value \texttt{true}, and \texttt{\#f} evaluates to the boolean value \texttt{false}.

            \subsubsection{Characters}
            Character values are 8-bit integers between 0 and 255, and follow ASCII standard.

        \subsection{Functions}
        Functional values are mappings from values to value.

        \subsection{Leaf}

        \subsection{N-Ary Tree Compound Type}
        A core feature of (g)ROOT is its n-ary tree compound value type. Every tree value consists of three components: an element, it's right-immediate sibling, and it's first child. This allows for the convenient implementation of robust recursive algorithms with an arbitrary branching factor.

        Every tree value in (g)ROOT may be either a full tree instance with an element, sibling, and child, or a leaf. The leaf value in (g)ROOT represents the nullary, or empty, tree.

        The tree type in (g)ROOT is modeled after a left-child right-sibling binary tree, where each node contains a reference to its first child and reference to its next sibling. This allows each node in Groot to have any number of children, while constraining the maximum number of fields per tree instance to three (element, sibling, and child).

        The tree type is very similar in usage to the one-dimensional list type present in many other functional languages, but enforces two additional invariants that empower programmers to shoot themselves in their feet less often.

        \begin{enumerate}
            \item The element of a tree instance may not be itself a tree. An element may be a value of any other type. This enforces a consistent structure among all trees that could be created in (g)ROOT.
            
            \textbf{Note}: it is possible to circumvent this requirement by wrapping a tree instance in a no-args lambda closure. This is a reasonable means of achieving nested data structures as it prevents the accidental creation of nested values; programmers who wrap tree instances in lambda closures likely did not with intention


            \item Every tree node must have a single immediate sibling and a single immediate child. This forces programmers to think in a purely recursive manner about their solutions.
            
            Trees may be conveniently constructed in-place using the following construction syntax:\\

\cbcolor{green}
\cbstart
\begin{lstlisting}
'(value left-tree right-tree)
\end{lstlisting}
\cbend

For example:\\

\cbcolor{green}
\cbstart
\begin{lstlisting}
'(0 leaf (1 (2  (3 leaf leaf) (4 (5 (6 (leaf leaf)) leaf))) (7 (8 leaf leaf) (9 (10 leaf leaf) leaf))))
\end{lstlisting}
\cbend

represents the following tree:
        \begin{figure}[h]
            \centering
            \includegraphics[width=0.9\linewidth]{images/LRMplanning.PNG}
        %     % \caption{} %% optional caption
            \label{fig:tree}
        \end{figure}

        \end{enumerate}

    \pagebreak
    %% End of Values %%%%%%%%%%%%%%%%%%%%%%%%%%%%%%%%%%%%%%%%%%%%%%%%%%%%%%



%%%%%%%%%%%%%%%%%%%%%%%%%%%%%%%%%%%%%%%%%%%%%%%%%%%%%%%%%%%%%%%%%%%%%%%%
\section{Names}
\label{sec:names}

    \subsection{Base Values}

    \subsection{Functions}

% \pagebreak
%% End of Names %%%%%%%%%%%%%%%%%%%%%%%%%%%%%%%%%%%%%%%%%%%



%%%%%%%%%%%%%%%%%%%%%%%%%%%%%%%%%%%%%%%%%%%%%%%%%%%%%%%%%%%%%%%%%%%%%%%%
\section{Constants}
\label{sec:const}


% \pagebreak
%% End of Constants %%%%%%%%%%%%%%%%%%%%%%%%%%%%%%%%%%%%%%%%%%%



%%%%%%%%%%%%%%%%%%%%%%%%%%%%%%%%%%%%%%%%%%%%%%%%%%%%%%%%%%%%%%%%%%%%%%%%
\section{Expressions}
\label{sec:expr}

        \hspace{1cm}\begin{minipage}[h]{1\textwidth}
        \begin{grammar}

            <expr> ::= $literal$
            \alt \hyperref[subsec:id]{$ident$}
            \alt \hyperref[subsec:op]{\emph{unary-operator}} $expr$
            \alt (\ \hyperref[subsec:op]{\emph{binary-operator}} $expr$ $expr$)
            \alt (\ \hyperref[subsec:id]{$ident$} \emph{expr-list}  )
            \alt (\ \texttt{val} \hyperref[subsec:id]{$ident$} $expr$  )
            \alt (\ \texttt{let} \hyperref[subsec:id]{$ident$} $expr$ $expr$  )
            \alt (\ \texttt{if} $expr$ $expr$ $expr$  )
            \alt (\ \texttt{lambda} (\ \{$arguments$\} )\ $expr$ )

            <literal> ::=  \hyperref[subsec:intlit]{\emph{integer-literal}} |  \hyperref[subsec:boollit]{\emph{boolean-literal}} | \hyperref[subsec:charlit]{\emph{character}} | \hyperref[subsec:key]{leaf}

            <arguments> ::= $\epsilon$
            \alt \hyperref[subsec:id]{$ident$} :: $arguments$
            

        \end{grammar}
        \end{minipage}

        \subsection{}

        \subsection{Lambda Expression}


% \pagebreak
%% End of Expressions %%%%%%%%%%%%%%%%%%%%%%%%%%%%%%%%%%%%%%%%%%%


    %%%%%%%%%%%%%%%%%%%%%%%%%%%%%%%%%%%%%%%%%%%%%%%%%%%%%%%%%%%%%%%%%%%%%%%%
    \section{Functions}
    \label{sec:func}


    % \pagebreak
    %% End of Functions %%%%%%%%%%%%%%%%%%%%%%%%%%%%%%%%%%%%%%%%%%%



\end{document}